\section{URLs}

URLs are a important part of the internet.
The acronym stands for {\bf Uniform Resource Locator}.
As you can see it gives an indication where a certain resource can be found.

\subsection{format}
An URL is formatted as the following:

\begin{figure}[h]
  \caption{Format of an URL.}
  \begin{verbatim}
    <scheme>://<auth>@<host>:<port>/<path>;<params>?<query>\#<frag>
  \end{verbatim}
\end{figure}

Let's talk about this, part for part.

\subsubsection{Scheme}
The {\bf scheme} indicates the protocol to use for the URL\@.
It will tell the operating system which program can process it.
The most common schemes used are {\it http://}, {\it https://} and {\it mailto://}.
This format explained is for HTTP and HTTPS\@.
An other scheme can use a different format.

\subsubsection{Auth}
This format is deprecated, but still used and accepted wildly.
It can provide a username and password for authentication.
Most of the times it is {\it base64} encoded.
The basic format is as following:

\begin{verbatim}
  username:password
\end{verbatim}

\subsubsection{Host}
This is the domain name most of the time.
It refers to the location of the server this is connected to.
If it's a domain name it'll get resolved by the {\bf DNS} first.

\subsubsection{Port}
Here you can override the TCP/IP port used to create a connection.
Normally this is 80 for HTTP, and 443 for HTTPS\@.

\subsubsection{Path}
This is the path where the resource is located.
THis indicates the location of the resource inside of the server.
An example is:

\begin{verbatim}
  /this/is/my/path.html
\end{verbatim}

\subsubsection{Params}
Here you can give more parameters to the url.
This is uncommon practice to use.
I recommend to use the {\it query} part instead.

\subsubsection{Query}
This is used to pass parameters via the URL\@.
These are simple key value parameters.

\begin{verbatim}
  ?foo=bar&hello=hi
\end{verbatim}

\subsubsection{Frag}
A frag is a location inside of a document.
It is most often used for referring to a specific chapter in a HTML document (an anchor).

\subsection{Punycode}
URLs were intended to only be used with ASCII characters.
This was fine in the beginning, as the internet was entirely in english.
In the modern day this is no longer the case, it's being used all over the world.
With other languages come other characters, characters which are not in ASCII\@.
To fix this issue punycode was invented.
It's a type of encoding to encode other characters with only ASCII\@.
For example:

\begin{verbatim}
  xn–pple-43d.com
\end{verbatim}

gives:
\begin{verbatim}
  apple.com
\end{verbatim}

\subsubsection{Phishing}
Punycode does make the browser vulnerable to a specific type of homograph attacks.
As you can see in last example the punycode renders as apple.com.
It is not actually apple.com, which can cause confusion.
You can even get SSL certificates for these domains, making is way more convincing.
