\section{Differentiation}

Differentiation is a really important concept of calculus.
With this you can calculate the slope of a function in a specific point.

\subsection{The product rule}
The product rule is a simple rule to differentiate two functions which are in a product.
The rule is as following:

\[(f * g)'(x)=f'(x) * g(x) + f(x) * g'(x)\]

\subsection{The quotient rule}
The quotient rule is a rule to differentiate a function which contains the quotient of two differentiable functions.
It works as following:

\[
  (\frac{f}{g})'(x) =
  \frac{g(x) * f'(x) - g'(x) * f(x)}{{g'(x)}^2}
\]

\subsection{The chain rule}
Some functions cannot be differentiated that easily.
You will need the chain rule for that.
Take the next function:

\[f(x) = \sqrt{x^2+1}\]
To apply the chain rule you need to substitute a part of the function with \(u(x)\).
You should choose a part which, when removed, makes the function way easier.
The previous function would look like this:

\[f(x) = \sqrt{u(x)}\]
\[u(x) = x^2+1\]
Now you can differentiate the functions separately.
This would look as following:

\[
  f'(x) =
  \frac{1}{2\sqrt{u(x)}} * u'(x) =
  \frac{1}{2\sqrt{x^2+1}} * 2x
\]
You can apply this to more than the square root, like power functions.
A more general rule for the chain rule is as following:

\[(f \circ g)'(x)=f'(g(x)) * g'(x)\]

\subsection{Implicit differentiation}
Till now we have assumed that we differentiate over a {\bf explicit function}.
This means it has the form of \(f(x) = y\).
But this is not always the case.
Take the following example:

\[x^3 + y^3 = 6xy\]
This is {\it not\/} an easy function to differentiate.
With implicit differentiation you differentiate both sides with respect to \(x\).
We can differentiate \(y^3\) with the chain rule.
So \(y^3\) will become \(3y^2 * \frac{dy}{dx}\).
Trying to do that with the example function gives you:

\[(x^3 + y^3)' = (6xy)'\]
\[3x^2 + 3y^2 \frac{dy}{dx} = 6y + 6x \frac{dy}{dx}\]
\[y^2 \frac{dy}{dx} - 2x \frac{dy}{dx} = 2y - x^2\]
\[\frac{dy}{dx}(y^2 - 2x) = 2y - x^2\]
\[\frac{dy}{dx} = \frac{2y - x^2}{y^2 - 2x}\]

\subsection{Linear approximation}
If you zoom into a function with the respective tangent line it\'ll start to look like that function.
With this you can approximate the value of a function with a tangent line close to that value.
The following is called the {\bf linearization} of \(f\) in \(a\):

\[l(x) = f(a) + f'(a)(x-a)\]
For example, if we want to calculate \(f(x) = \sqrt{x}\) with \(x = 4.36\).
We take an \(a\) we know close to that point, \(4\) would make sense.
This gives the following function:

\[l(x) = 2 + \frac{1}{4}(x - 4) = 1 + \frac{x}{4}\]
Now we can fill in our \(x\) into \(l(x)\).

\[l(4.36) = 1 + \frac{4.36}{4} = 1 + \frac{109}{100} = 2.09 \approx \sqrt{4.36}\]
